\documentclass[11pt, a4paper]{article}
\usepackage[inner=1in,outer=1in,top=1in,bottom=1in]{geometry}
\pagestyle{empty}
\usepackage{placeins}
\usepackage{graphicx}
\usepackage{fancyhdr, lastpage, bbding, pmboxdraw}
\usepackage{amsmath, amssymb}
\usepackage[usenames,dvipsnames]{color}
\definecolor{darkblue}{rgb}{0,0,.6}
\definecolor{darkred}{rgb}{.7,0,0}
\definecolor{darkgreen}{rgb}{0,.6,0}
\definecolor{red}{rgb}{.98,0,0}
\usepackage[colorlinks,pdfusetitle,urlcolor=darkblue,citecolor=darkblue,linkcolor=darkred,bookmarksnumbered,plainpages=false]{hyperref}
%\renewcommand{\thefootnote}{\fnsymbol{footnote}}

\pagestyle{fancyplain}
\fancyhf{}
\lhead{ \fancyplain{}{\CourseTitle} }
%\chead{ \fancyplain{}{} }
\rhead{ \fancyplain{}{\CourseSemester \CourseYear} }
%\rfoot{\fancyplain{}{page \thepage\ of \pageref{LastPage}}}
\fancyfoot[RO, LE] {page \thepage\ of \pageref{LastPage} }
\thispagestyle{plain}
\usepackage{tabularx}


%%%%%%%%%%%%%%%%%%%%%%%%%%%%%%%%%%%%
\usepackage{xspace}

\newcommand{\CourseNumber}{NPRE247}
\newcommand{\CourseTitle}{Modeling Nuclear Energy Systems\xspace}%
\newcommand{\CourseInstructor}{Dr. Madicken Munk\xspace}%
\newcommand{\CourseSemester}{Spring\xspace}%
\newcommand{\CourseYear}{2023\xspace}%
\newcommand{\CourseDays}{MWF\xspace}%
\newcommand{\CourseStart}{10:00\xspace}%
\newcommand{\CourseEnd}{10:50\xspace}%
\newcommand{\CourseInstructorEmail}{mmunk2@illinois.edu}
\newcommand{\CourseRoom}{3038\xspace}%
\newcommand{\CourseBuilding}{Campus Instructional Facility\xspace}%
\newcommand{\CourseZoom}{https://illinois.zoom.us/j/85997782066?pwd=WnBwaXBJU24vSmxaL3FqUEtNWkhnZz09}%
\newcommand{\CourseUniversity}{University of Illinois, Urbana-Champaign\xspace}%
\newcommand{\TeachingAssistanta}{Arthur Mazzeo \xspace}%
\newcommand{\TAaOfficeHourDays}{Date Unknown \xspace}%
\newcommand{\TAaOfficeHourStart}{TBA\xspace}%
\newcommand{\TAaOfficeHourEnd}{TBA\xspace}%
\newcommand{\TAaOfficeHourPlace}{the 220 Talbot Laboratory Student Lounge\xspace}
\newcommand{\TeachingAssistantb}{Ceser Zambrano \xspace}%
\newcommand{\TAbOfficeHourDays}{Date Unknown \xspace}%
\newcommand{\TAbOfficeHourStart}{TBA\xspace}%
\newcommand{\TAbOfficeHourEnd}{TBA\xspace}%
\newcommand{\TAbOfficeHourPlace}{the 220 Talbot Laboratory Student Lounge\xspace}
\newcommand{\CourseAssistants}{Brady\xspace}%
\newcommand{\CAOfficeHours}{Tu/Th 1-2pm and W/F 12-1pm \xspace}%
%\newcommand{\CAOfficeHourDays}{Tuesdays, Wednesdays, and Thursdays\xspace}%
%\newcommand{\CAOfficeHourStart}{1:00pm\xspace}%
%\newcommand{\CAOfficeHourEnd}{3:00pm\xspace}%
\newcommand{\CAOfficeHourPlace}{the 131 Talbot Laboratory Computer Lab\xspace}
\newcommand{\MunkOfficeHourDays}{Fridays\xspace}%
\newcommand{\MunkOfficeHourStart}{11:00 a.m.\xspace}%
\newcommand{\MunkOfficeHourEnd}{11:50 a.m.\xspace}%
\newcommand{\MunkOfficeHourPlace}{118 Talbot Laboratory, 104 S. Wright St.\xspace}
%\newcommand{\Course<++>}{<++>}
%\newcommand{\Course<++>}{<++>}
%%%%%%%%%%%%%%%%%%%%%%%%%%%%%%%%%%%%
\title{\CourseNumber: \CourseTitle\\}
\author{\CourseUniversity}
\date{\CourseSemester \CourseYear}
\begin{document}
\maketitle
%\setlength{\unitlength}{1in}
\renewcommand{\arraystretch}{1.5}
\begin{center}
\begin{table}[h]
\begin{tabularx}{\textwidth}{rXrX}
\hline
\textbf{Instructor:} & \CourseInstructor & \textbf{Time:} & \CourseDays \CourseStart -- \CourseEnd \\
\textbf{Email:} &  \href{mailto:\CourseInstructorEmail}{\CourseInstructorEmail} & \textbf{Place:} & \CourseRoom \CourseBuilding\\
\textbf{Course Zoom:} & \url{\CourseZoom} & & \\
\hline
\end{tabularx}
\end{table}
\end{center}

\paragraph{Course Pages:}
\begin{enumerate}
        \item \url{https://canvas.illinois.edu/courses/35562}
        \item \url{https://www.gradescope.com/courses/497029}
        \item \url{https://github.com/munkm/npre247}
        \item \url{https://mybinder.org/v2/gh/munkm/npre247/23-spring}
\end{enumerate}

\paragraph{Course Zoom:} A persistent zoom link (\url{\CourseZoom})
for our course is included in
the syllabus and on the course Canvas page.
This zoom will only be active when campus
requires that we have remote instruction (e.g. the first week of classes this
semester). It will not be active if there is in person instruction unless Dr.
Munk announces it later in the semester.

\paragraph{TA Office Hours:} The teaching assistants for the course will hold
office hours in \TAaOfficeHourPlace.
\TeachingAssistanta will hold office hours \TAaOfficeHourDays from
\TAaOfficeHourStart to \TAaOfficeHourEnd and
\TeachingAssistantb will hold office hours \TAbOfficeHourDays from
\TAbOfficeHourStart to \TAbOfficeHourEnd.
Grade disputes will not be addressed in TA office hours.

\paragraph{Office Hours:} Dr. Munk  will hold office hours on
\MunkOfficeHourDays from \MunkOfficeHourStart to \MunkOfficeHourEnd in
\MunkOfficeHourPlace Supplemental office hours are by appointment only
and will never be available with less than 24 hours notice.
Before making an appointment, please try the following options:
\begin{itemize}
\item If your colleagues might be helpful, please post your questions in the
        canvas discussion forum provided for this purpose.
\item If the TAs might be helpful, please attend their office hours.
\item Email Dr. Munk. If possible, please phrase your question such that it
        can be answered `Yes' or `No'.  Questions which require substantial
        response should be asked during the lecture or office hours.
\end{itemize}

If none of the above are successful or appropriate, you may email me with a
selection of times of your availability and we can find a time that is mutually
agreeable.

\paragraph{Main References:}
A few essential references for this course will be assigned as readings. The
required text for this course is \cite{shultis_fundamentals_2016} while
\cite{lamarsh_introduction_2017} is also recommended. For assistance in
computational projects, \cite{scopatz_effective_2015} is available online and
in the library as an ebook.

\bibliographystyle{unsrt}
\renewcommand{\refname}{\normalfont\selectfont\normalsize}\vspace{-1cm}
\bibliography{bibliography}

\paragraph{Objectives:}

This course will equip students to:

\begin{itemize}
\item Apply elementary nuclear physics to nuclear engineering.
\item Classify and compare nuclear reactor materials and components.
\item Evaluate steady-state and transient operation of nuclear reactors.
\item Calculate aspects of nuclear energy removal and conversion.
\item Select and simulate radiation shielding.
\end{itemize}

\paragraph{Prerequisites:}
\begin{itemize}
\item PHYS 212
\item MATH 285 (credit or concurrent registration)
\end{itemize}

\paragraph{Grading Policy:} Grades will be assigned as a weighted sum of the following work.

\begin{table}[h]
\begin{tabularx}{\textwidth}{Xr}
\textbf{Work} & \textbf{Weight}\\
\hline
\textbf{Quizzes} & (10\%) \\
\textbf{Homework} & (15\%) \\
\textbf{Computer Projects} & (30\%) \\
\textbf{Midterm 1} & (15\%) \\
\textbf{Midterm 2} & (15\%) \\
\textbf{Final Exam} & (15\%) \\
\hline
\textbf{Total} & (100\%) \\
\end{tabularx}
\end{table}

\paragraph{Important Dates:}
\begin{center} \begin{minipage}{3.8in}
\begin{flushleft}
Midterm \#1       \dotfill February 13, 2022, 10:00am  \\
Midterm \#2       \dotfill March 31, 2022, 10:00am  \\
CP1 Deadline      \dotfill TBD, 11:59pm  \\
CP2 Deadline      \dotfill TBD, 11:59pm  \\
CP3 Deadline      \dotfill TBD, 11:59pm  \\
Final Exam        \dotfill Month, Day, Year, Time  \\
\end{flushleft}
\end{minipage}
\end{center}

\paragraph{Class Policies:}

\begin{itemize}
\item[] \textbf{Integrity:} This is an institution of higher
learning. You will be swiftly ejected from the course if you are caught
undermining its integrity. Note the
\href{http://www.provost.illinois.edu/academicintegrity/students.html}{Student's
Quick Reference Guide to Academic Integrity} and the
\href{http://studentcode.illinois.edu/article1_part4_1-401.html}{Academic
Integrity Policy and Procedure}.
\item[] \textbf{Attendance:} Regular attendance is expected. Request approval for absence for extenuating circumstances prior to absence.
\item[] \textbf{Electronics:} Active participation is essential and expected.
        Accordingly, students must turn off all electronic devices (laptop,
        tablets, cellphones, etc.) during class. Exceptions may be granted for
        laptops and tablets if engaging in computational exercises or taking notes.
\item[] \textbf{Collaboration:} Collaboratively reviewing course materials and
  studying for exams with fellow students can be enriching.  This is
  recommended.  However, unless otherwise instructed, homework assignments are
  to be completed independently and materials submitted as homework should be
  the result of one's own independent work. Dr. Munk recommends working through
  the problems independently and then checking work with peers. Explaining your
  process is a good exercise to retain course material.
\item[] \textbf{Late Work:} Late work will not be accepted. Plan ahead.
        We will drop the lowest scoring homework and quiz grade for each
        student. This shall accomodate unforseen circumstances that may
        contribute to a missed quiz or homework assignment.

\item[] \textbf{Make-up Work:} There will be no negotiation about late work
        except in the case of absence documented by an absence letter from the
        Dean of Students.  The university policy for requesting such a letter
        is in
        \href{http://studentcode.illinois.edu/article1_part5_1-501.html}{the
        Student Code}. Please note that such a letter is appropriate for many
        types of conflicts, but that religious conflicts require special early
        handling. In accordance with university policy, students seeking an
        excused absence for religious reasons should complete the
	\href{http://odos.illinois.edu/community-of-care/resources/students/religious-observances/}{Request for Accommodation for Religious Observances Form}
        The student should submit this
        form to the instructor and the Office of the Dean of Students by the end of the
        second week of the course to which it applies.


\item[] \textbf{Grade Disputes:} It is important that you understand and agree
        with the grade you receive on assignments and exams. If you would like
        to dispute your score, you must send an explanation by email to Dr.
        Munk within one week of recieving the grade.
        \textbf{Do not expect us to regrade anything while in conversation with
        you} as that would not be fair to the other students in the class, whose
        homeworks were graded without them present.  If you request a regrade,
        be aware that the it is possible that your score will go down.
        Regrade requests should be based on an error on our part (e.g., adding
        up the points incorrectly) or what you suspect is a misunderstanding of
        your work (e.g., arriving at the correct answer using an unexpected
        technique). Regrade requests that argue with the rubric (e.g., ``this is
        wrong, but you took too many points off'') will be returned without
        consideration.
        \textbf{Your work should stand alone.} If an assignment is disorganized or
        ambiguous, and requires an extensive explanation to the grader, you
        will likely still lose points. The homeworks not only evaluate your
        understanding of the material - they also evaluate your ability to
        communicate that understanding clearly and concisely.
\end{itemize}

\paragraph{Accessibility:} I hope that this course will be inclusive and
accommodating for all learners. As such, I am committed upholding the vision
and values of \href{http://www.inclusiveillinois.illinois.edu/index.html}{Inclusive Illinois}
in my
classroom.  With regard to accommodating all learners, please note that many
resources are provided through
\href{http://disability.illinois.edu/academic-support/accommodations}{the
Division of Disability Resources and Educational Services}.  To request
particular accommodations, please contact me as soon as possible so that we can
work out any necessary arrangements.

\paragraph{Safety:}
Emergencies can happen anywhere and at any time, so it’s important that we take
a minute to prepare for a situation in which our safety could depend on our
ability to react quickly. Take a moment to learn the different ways to leave
this building. If there's ever a fire alarm or something like that, you’ll know
how to get out and you'll be able to help others get out. Next, figure out the
best place to go in case of severe weather - we'll need to go to a low-level in
the middle of the building, away from windows. And finally, if there's ever
someone trying to hurt us, our best option is to run out of the building. If we
cannot do that safely, we'll want to hide somewhere we can't be seen, and we'll
have to lock or barricade the door if possible and be as quiet as we can. We
will not leave that safe area until we get an Illini-Alert confirming that it's
safe to do so. If we can't run or hide, we'll fight back with whatever we can
get our hands on. If you want to better prepare yourself for any of these
situations, visit \url{police.illinois.edu/safe}. Remember you can sign up for
emergency text messages at \url{emergency.illinois.edu}. This
\href{http://police.illinois.edu/dpsapp/wp-content/uploads/2017/08/syllabus-attachment.pdf}{one-page-handout}
discusses the Illinois Run-Hide-Fight strategy.


\paragraph{Other Resources:}
University students typically experience a wide range of stressors during their
time on campus. Accordingly, campus resources exist to help students manage
stress levels, mental health, physical health, and emergencies while navigating
this environment. I hope you will take advantage of these campus resources.

\begin{itemize}
\item \href{https://campusrec.illinois.edu/}{The Campus Recreational Centers}
\item \href{http://counselingcenter.illinois.edu/}{The Counselling Center}
\item \href{http://www.mckinley.illinois.edu/clinics/mental\_health.htm}{The McKinley Mental Health Clinic}
\item \href{http://odos.illinois.edu/emergency/}{The Emergency Dean}
\end{itemize}

\pagebreak
\FloatBarrier
\renewcommand{\arraystretch}{1}
\begin{table}[h]
\begin{center}
\begin{tabular}{lllcccccc}
\multicolumn{8}{c}{\textbf{Course Schedule:}\textit{ Note that this schedule is
subject to change.}}\\
&&&&&&&&\\
\textbf{Date} & \textbf{Week} & \textbf{Day} & \textbf{Unit} & \textbf{Chap.} & \textbf{Quiz}& \textbf{Quiz} & \textbf{HW} & \textbf{HW}\\
              &  &  &  &  & \textbf{Given} & \textbf{Due} & \textbf{Given} & \textbf{Due}\\ \hline
\hline
1/18 & 1 & W & Intro              &  & 1 &  &  & \\
1/20 & 1 & F & Fundamentals       & 1 & 2 & 1 & 1 &  \\
1/23 & 2 & M & Modern Physics     & 2 &   & 2 & &  \\
1/25 & 2 & W & Modern Physics     & 2 &   &  &  &  \\
1/27 & 2 & F & Nuclear Models     & 3 & 3 &  & 2 & 1  \\
1/30 & 3 & M & Nuclear Energetics & 4 &   & 3 & &  \\
2/1 & 3 & W & Nuclear Energetics  & 4 &   &  &  &  \\
2/3 & 3 & F & Radioactivity       & 5 & 4 &  & 3 & 2 \\
2/6 & 4 & M & Radioactivity       & 5 &   & 4 &  &  \\
2/8 & 4 & W & Radioactivity       & 5 &   &  &  &  \\
2/10 & 4 & F & Review             &   & 5 &  & 4 & 3 \\
2/13 & 5 & M & Exam 1             &   &   & 5  &  &  \\
2/15 & 5 & W & Radioactivity  & 6 &   &  &  &  \\
2/17 & 5 & F & Exam I Review  & 6 & 6 &  & 5 & 4 \\
2/20 & 6 & M & Intro to Python  & 6 &   & 6 &  &  \\
2/22 & 6 & W & Binary Nuc. Rxns.      &  &  &  &  &  \\
2/24 & 6 & F & Binary Nuc. Rxns.  & 6 & 7 &  & 6 & 5  \\
2/27 & 7 & M & Binary Nuc. Rxns. &  &  & 7 &  &  \\
3/1 & 7 & W & Rad. Int. w/Matter  & 6 &  &  &  &  \\
3/3 & 7 & F & Rad. Int. w/Matter  & 7 & 8 &  & 7 & 6 \\
3/6 & 8 & M & Neutron Balance     & 7 &  & 8 &  &  \\
3/8 & 8 & W & Neutron Balance         & 9 &   &  &  &  \\
3/10 & 8 & F & Criticality        & 10 & 9 &  & 8 & 7 \\
3/13 & 9 & M & \textbf{Spring Break}&  &  &  &  &  \\
3/15 & 9 & W & \textbf{Spring Break}&  &  &  &  &  \\
3/17 & 9 & F & \textbf{Spring Break}&  &  &  &  &  \\
3/20 & 10 & M & 6 factor          & 10 &  & 9 &  &  \\
3/22 & 10 & W & 6 factor          & 10 &  &  &  &  \\
3/24 & 10 & F & Exam 2          & 10 & 10 &  & 9 & 8 \\
3/27 & 11 & M & 6 Factor            & 10 &  & 10 &  &  \\
3/29 & 11 & W & Criticality            & 10 &  &  &  & \\
3/31 & 11 & F & Nuclear Reactors  &  &  11 &  & 10 & 9 \\
4/3 & 12 & M & Diffusion &  10  &   & 11 &  &  \\
4/5 & 12 & W & Diffusion & 10  &  & 10 &  &  \\
4/7 & 12 & F & Diffusion & 10 & 12 &  & 11 & 10 \\
4/10 & 13 & M & Diffusion & 10 &  & 12 & &  \\
4/12 & 13 & W & Nuclear Fuel Cycles & 10 &  &  &  &  \\
4/14 & 13 & F & \textbf{Student Conference}    & 10 & 13 &  & 12 & 11 \\
4/17 & 14 & M & Exam 2 Review    &  &  & 13  &  &  \\
4/19 & 14 & W & Ethics     & 10 & &  &  & \\
4/21 & 14 & F & Ethics    & 10 & 14 &  & 13 & 12 \\
4/24 & 15 & M & Reactivity Feedback    & 10  &  & 14 & &  \\
4/26 & 15 & W & Fission Prod. Poison        & 11 &  & 14 &  &  \\
4/28 & 15 & F & Fission Prod. Poison        & 11 &  &  & 14 & 13 \\
5/1 & 16 & M & Nuclear Power                &  &  & &  &      \\
5/3 & 16 & W &  Review               &  &  &  &  &  14 \\
TBD & -- & -- & \textbf{Final Exam}  &  &  &  &  &     \\
\end{tabular}
\end{center}
\end{table}
\FloatBarrier



%%%%%% THE END
\end{document}
